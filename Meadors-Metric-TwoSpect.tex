\documentclass{article}
\begin{document}
\title{First approximation to the TwoSpect metric \\ 
LIGO-T1600318}
\author{Grant David Meadors}
\date{\today}
%\affiliation{AEI Hannover/Golm}

\maketitle

\section{Introduction}

TwoSpect papers by Goetz \& Riles [\textit{Classical and Quantum Gravity} \textbf{28} (2011) 215006] and Meadors, Goetz \& Riles [\textit{Classical and Quantum Gravity} \textbf{33} (2011) 105017] have not explicitly stated the mismatch metric for the $R$ statistic.
This omission stems from complexity: $R$ is a normalized, background-subtracted sum of powers of FFT pixels of the powers of FFT bins.
In the first methods papers, we have relied on Monte Carlo-derived approximations to the mismatch and inferred template grids spacings of $\delta f = \delta \Delta f = (2 T_\mathrm{sft})^{-1}$.
(In the S6 Scorpius X-1 search, the $\delta \Delta f$ grid was twice as dense).
We would like to make a first approximation to the TwoSpect metric, analytically, as a sanity check to these simulations.


\section{Derivation}

The metric is computed from the mismatch in the $R$ statistic, which is the sum of pixel powers $|\mathcal{F}(|x_z (n)|^2)|^2$.
Start with the 2011 TwoSpect methods paper, eq. 20, giving the pixel power in the 1st FFT plane as the Dirichlet kernel:

\begin{equation}
|x_z (n)|^2 = \frac{2}{3} \frac{\mathrm{sinc}^2 (k+ \Delta f T_\mathrm{sft} \sin (2 \pi T_m n / P)-z}{[(k+df T_\mathrm{sft}
              \sin (2 \pi T_m n / P) -z)^2 -1]^2},
\end{equation}

\noindent and let $\epsilon = k+ df0 T_\mathrm{sft} \sin(2\pi T_m n / P)-z$,
ergo $\epsilon = T_\mathrm{sft}(f0+ df0 \sin(2\pi T_m n /P) - (f+df) )$.
Note $T_m = T_\mathrm{sft/2}$. 
We will drop the coefficient $2/3$, because we will take the ratio of $R$ statistics (and thus these pixel powers).
We use $\mathrm{sinc}(x) \equiv \sin(\pi x)/(\pi x)$,
so

\begin{equation}
|x_z (n)|^2 = \mathrm{sinc}^2(\epsilon)/[\epsilon^2-1]^2.
\end{equation}

Suppose that a 2nd FFT plane pixel is proportional to $(|x_z(n)|^2)^2$. This is the
big assumption: the true pixel is the power of a Fourier transform of
$|x_z(n)|^2$. The effect of a proper treatment would be to make the
`inside' shallower than the outside; compare Bessel function horns.
Since the projected phase component and the Dirichlet kernel itself move
quadratically in time, we may be justified for now in ignoring this
higher-order treatment. (For orbital period, one cannot). So pretend the
2nd FFT pixel has the shape of the Dirichlet kernel.

Taylor expand that pixel, \textit{i.e.,} Dirichlet kernel squared:

\begin{eqnarray}
|x_z (n)|^4 & \approx & 1 + \left(2- \frac{\pi^2}{3}\right) \epsilon^2,\\
             &=& 1 + \left(2 -\frac{\pi^2}{3}\right) T_\mathrm{sft}^2
                 (f_0+ \Delta f_0 \sin(2\pi T_m n /P) - (f+\Delta f))^2.
\end{eqnarray}

Remember that for the 2nd FFT plane power that contributes to the $R$ statistic, most power is concentrated in pixels at the modulation depth extremes, especially at the first harmonic.
Ignoring background subtraction, we assume a mismatch will affect all the pixels in these extremes equivalently, regardless of weighting.
Moreover, we make assume that the template weight normalization in the denominator of $R$ is the same at the true position and mismatch, so that it cancels out.
Then construct this highly-simplified $R$ statistic out of two of these pixels, at upper and
lower extremes:

\begin{equation}
R \propto 2 + \left(2 -\frac{\pi^2}{3}\right) T_\mathrm{sft}^2
             [(f_0+\Delta f_0- (f+\Delta f))^2 + (f_0-\Delta f_0- (f-\Delta f))^2].
\end{equation}

\noindent
The maximum is when $f=f_0$, $\Delta f = \Delta f_0$,

\begin{equation}
R_\mathrm{max} = 2.
\end{equation}

\noindent So 

\begin{eqnarray}
\mu &=& 1 - R / R_\mathrm{max},\\
      &=& \left(\frac{\pi^2}{6} - 1\right) T_\mathrm{sft}^2 \nonumber \\
      &~& \times [(f_0+\Delta f_0- (f+\Delta f))^2 + (f_0-\Delta f_0- (f- \Delta f))^2].
\end{eqnarray}

\noindent Then the metric follows from
Brady \textit{et al}, [\textit{Physical Review D} \textbf{57} (1998) 2101], eq 5.12.

\begin{equation}
g_{ab} = \frac{1}{2} \partial_a \partial_b \mu.
\end{equation}

\noindent Begin with

\begin{eqnarray}
\partial_f (\mu) &=& 4 \left(\frac{\pi^2}{6} -1\right) T_\mathrm{sft}^2 (f-f_0),\\
\partial_{\Delta f} (\mu) &=& 4 \left(\frac{\pi^2}{6}-1\right) T_\mathrm{sft}^2 (\Delta f - \Delta f_0).
\end{eqnarray}

\noindent and obtain the result.

\section{Metric result}

For the $[f, \Delta f]$ parameter space, the mismatch metric For TwoSpect is, in this first approximation,

\begin{equation}
  \mathsf{g} = 2 \left(\frac{\pi^2}{6} - 1\right) T_\mathrm{sft}^2  
  \left[ 
  \begin{array}{c c}
  1 & 0\\
  0 & 1
  \end{array}
  \right]
\end{equation}

Along lines of $f$ or $\Delta f$, this returns
the same mismatch using Brady \textit{et al} eq 5.11 ($\mu = g_{\alpha\beta} \Delta \lambda^\alpha \Delta \lambda^\beta$) as does the actual mismatch:

\begin{equation}
\mu = (\pi^2/6 -1) T_\mathrm{sft}^2 
        [(f_0 + \Delta f_0 - (f + \Delta f))^2 + (f_0 - \Delta f_0 - (f - \Delta f))^2 ],
\end{equation}

Of course, the metric
fails to capture the degeneracies along the diagonals.
As a quadratic approximation, it cannot capture that complexity.

\section{Grid spacing}

We find that the nearest template in either dimension should be

\begin{equation}
[f-f0] = \left(2 (\pi^2/6 - 1)\right)^{-1/2} T_\mathrm{sft}^{-1} \sqrt{\mu}
       \approx 0.88  T_\mathrm{sft}^{-1} \sqrt{\mu}
\end{equation}

\noindent
which for $0.2$ mismatch (our standard in S6) is

\begin{equation}
[f-f0] = 0.39 T_\mathrm{sft}^{-1}
\end{equation}

We have a grid spacing of $0.5/T_\mathrm{sft}$, meaning that in the 1D case the
nearest template should be no farther away than $0.25/T_\mathrm{sft}$.

Hence, our choice of grid-spacing seems analytically justified.


\end{document}
